\documentclass[professionalfont]{beamer}
\usepackage{newtxtext,newtxmath}
%Information to be included in the title page:
%\begin{frame}
%\frametitle{}
%\end{frame}
%\begin{frame}[fragile]
  %\begin{lstlisting}
       %%CODE HERE
   %\end{lstlisting}
%\end{frame}

\usepackage{listings}
\usepackage{xcolor}

\definecolor{codegreen}{rgb}{0,0.6,0}
\definecolor{codegray}{rgb}{0.5,0.5,0.5}
\definecolor{codepurple}{rgb}{0.58,0,0.82}
\definecolor{backcolour}{rgb}{0.95,0.95,0.92}

\lstdefinestyle{mystyle}{
    backgroundcolor=\color{backcolour},   
    commentstyle=\color{codegreen},
    keywordstyle=\color{magenta},
    numberstyle=\tiny\color{codegray},
    stringstyle=\color{codepurple},
    basicstyle=\ttfamily\footnotesize,
    breakatwhitespace=false,         
    breaklines=true,                 
    captionpos=b,                    
    keepspaces=true,                 
    numbers=left,                    
    numbersep=5pt,                  
    showspaces=false,                
    showstringspaces=false,
    showtabs=false,                  
    tabsize=2
}

\lstset{style=mystyle}
\title{Object Oriented design }
\author{Mackenzie Norman}



\begin{document}

\frame{\titlepage}

\begin{frame}
\frametitle{}
``Object - oriented approaches localize Information around objects''

        - \textit{Edward V. Berard}\newline

\end{frame}
\begin{frame}
\frametitle{Why?}
Object Oriented design allows for
\begin{list}{- }{}
    \item Greater extensibility
    
    \item Less repeated Code
    \item Easier design process
\end{list}

\end{frame}

\begin{frame}
\frametitle{Core Concepts of Object Oriented Design}
There are key three features of Object Oriented design. 

\begin{list}{- }{}
    \item Encapsulation \& data-hiding
    \item Inheritance
    \item Polymorphism 
\end{list}

\textit{Note: In small software projects encapsulation is often less important than the other 2, so I will most likely ignore it in this presentation, if you are writing large software in R please find a new profession}

\end{frame}


\begin{frame}
    \frametitle{Polymorphism and Inheritance}
    Polymorphism and Inheritance are what make object oriented programming so powerful. Lets use the example of a car. We can think of some other types of cars:
\begin{list}{- }{}
    \item A Sedan
    \item A Truck
    \item A hatchback
\end{list}

We inherently know that these \textit{objects} are all types of cars and will preform the same activity (driving)
However - these vehicles have very different engines \textit{under the hood}. 
Object Oriented Code is an attempt to make the future code writer,  the driver. It doesn't matter how the car drives or what the engine does, a driver can drive all three of these cars

\end{frame}

\begin{frame}[fragile]
    \frametitle{Polymorphism and Inheritance}
    Continuing the analogy, polymorphism is the idea that road shouldn't handle a sedan differently from a truck. \newline

    In terms of code we can see this like:
    \begin{lstlisting}[language=R]
        s <- sedan$new()
        t <- truck$new()

        #these should do the same thing
        s$drive()
        t$drive()
    \end{lstlisting}

\end{frame}
\begin{frame}
    \frametitle{Polymorphism and Inheritance}
    While polymorphism is fairly abstract and able to be understood without code, Inheritance is the mechanism in which polymorphism is realized. \newline
    Unfortunately (or fortunately) to understand Inheritance we need to look at code

\end{frame}


\begin{frame}[fragile]
    \frametitle{Polymorphism and Inheritance}
    Here is our pet class in code, its very simple, just a name and a favorite food
 \lstinputlisting[language=R, firstline=2, lastline=12]{cats_in_r6pt1.r}
\end{frame}
\begin{frame}[fragile]
    \frametitle{Polymorphism and Inheritance}
    Now we can create a cat class that \textit{inherits} from this pet class.

    This relationship is often referred to parent and child classes.

 \lstinputlisting[language=R, firstline=19, lastline=30]{cats_in_r6pt1.r}
\end{frame}

\begin{frame}[fragile]
    \frametitle{Polymorphism and Inheritance}

 \lstinputlisting[language=R, firstline=33, lastline=49]{cats_in_r6pt1.r}
\end{frame}

\begin{frame}[fragile]
    \frametitle{Polymorphism and Inheritance}

 \lstinputlisting[language=R, firstline=51, lastline=64]{cats_in_r6pt1.r}
\end{frame}
\begin{frame}[fragile]
    \frametitle{Polymorphism and Inheritance}

 \lstinputlisting[language=R, firstline=66, lastline=83]{cats_in_r6pt1.r}
\end{frame}
\end{document}